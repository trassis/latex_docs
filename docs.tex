\documentclass[10pt]{article}

\usepackage[brazil]{babel} % Portugues
\usepackage[utf8]{inputenc} % Formatation
\usepackage[a4paper,margin=2cm]{geometry} %Sets the page geometry
\usepackage{amsmath,amssymb,amsfonts,amsthm} % Maths
\usepackage{graphics}
\usepackage{import}
\usepackage{multicol}

\setlength\parindent{0pt} % Sem indentação em novo parágrafo
\selectlanguage{brazil} 
\linespread{1.0} % Espaço entre linhas

\begin{document}


% Tabela phi
\resizebox{0.95\textwidth}{!}{%
\import{./}{table1.tex}
}

\pagebreak

\begin{multicols*}{2}

\section{Variáveis Aleatorias}

Função Distribuição de Probabilidade: 
\[ 
    f(x) = P(X \leq x)  
\]
Valor Esperado:
\begin{align*}
    & E(x) = \sum_x xp(x) 
    \\
    & E(aX + b) = aE(x) + b 
    \\
    & E(X^n) = \sum_x x^n p(x)
\end{align*}
Variância:
    $Var(x)=E(X-\mu )^2$

\section{Distribuições Discretas}

\textbf{Binomial:} $ X \sim B(n,p) $ representa o número de sucessos em $ n $ tentativas independentes,
cada qual com probabilidade de sucesso $ p $
\begin{gather*}
    P(X = i) = \binom{n}{i} p^i (1-p)^{n-i}
    \\
    E[X^k] = npE[(Y+1)^{k-1}] \text{ sendo } Y \sim B(n-1,p)
    \\
    Var[X] = np(1-p)
\end{gather*}

\textbf{Bernoulli:} $ X \sim Be(p) $ é uma varável aleatória binominal com n=1
\begin{gather*}
    P(X = i) = p^i (1-p)^{n-i}
    \\
    E[X^k] = p
    \\
    Var[X] = p(1-p)
\end{gather*}

\textbf{Poisson:} $ X \sim P(\lambda) $ assume valores $ i $ para $ 0,1,2,... $, sendo 
$ \lambda > 0 $
\begin{gather*}
    P(X = i) = e^{-\lambda} \frac{\lambda^i}{i!}
    \\
    E[X] = Var[X] = \lambda
\end{gather*}
    - Aproximação da distribuição binomial: se $ \lambda = np $, então 
    $ B(n,p) \approx P(\lambda) $
\\

\textbf{Binominal negativa:} define o número de tentativas necessárias para 
obter r sucessos
\begin{gather*}
    P(X_r = n) = \binom{n-1}{r-1} p^r (1-p)^{n-r}
    \\
    E(X)
    \\
    V(X)
\end{gather*}

\textbf{Geométrica:} $ X\sim Geo(p) $ define o número de tentativas necessárias 
para obter um sucesso. É uma variável binominal não negativa para r=1
\begin{gather*}
   P(X = i) = (1-p)^{n-1}p
   \\
   E(X)
   \\
   V(X)
\end{gather*}
\textbf{Hipergeométrica:} define o número de um mesmo evento ocorrer em um espaço
amostral sem repetição
\begin{gather*}
   P(X = i) = \binom{n}{i}p^{i}(1-p)^{n-i}
   \\
   E(X)
   \\
   V(X)
\end{gather*}

\section{Distribuições Continuas}
Falta aqui 

\end{multicols*}

\end{document}